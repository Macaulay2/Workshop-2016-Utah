\documentclass[11pt,fullpage]{amsart}
%   \overfullrule=5pt
\usepackage[active]{srcltx}
\usepackage[all]{xy}
\usepackage{stmaryrd}
\usepackage{calc,amssymb,amsthm,amsmath}
\usepackage{fullpage}
\RequirePackage[dvipsnames,usenames]{color}
\let\mffont=\sf
%%%   Setup to use Ralph Smith Formal Script font:
%%   Provided by Sandor Kovacs to Karl Schwede, modified further
\DeclareFontFamily{OMS}{rsfs}{\skewchar\font'60}
\DeclareFontShape{OMS}{rsfs}{m}{n}{<-5>rsfs5 <5-7>rsfs7 <7->rsfs10 }{}
\DeclareSymbolFont{rsfs}{OMS}{rsfs}{m}{n}
\DeclareSymbolFontAlphabet{\scr}{rsfs}

\newenvironment{pf}[1][{\it Proof:}]{\begin{trivlist}
\item[\hskip \labelsep {\bfseries #1}]}{\end{trivlist}}

%%***Collaborative editing commands***
\newcommand{\note}[1]{\marginpar{\sffamily\tiny #1}}

\def\todo#1{\textcolor{Mahogany}%
{\sffamily\footnotesize\newline{\color{Mahogany}\fbox{\parbox{\textwidth-15pt}{#1}}}\newline}}


%%***Common Names***
\newcommand{\Cech}{{$\check{\text{C}}$ech} }
\newcommand{\mustata}{Musta{\c{t}}{\u{a}}}
\newcommand{\etale}{{\'e}tale }


%%%************Functors, derived categories and related****************
\newcommand{\myR}{{\bf R}}
\newcommand{\myH}{h}
\newcommand{\myL}{{\bf L}}
\newcommand{\qis}{\simeq_{\text{qis}}}
\newcommand{\tensor}{\otimes}
\newcommand{\hypertensor}{{\uuline \tensor}}
\newcommand{\mydot}{{{\,\begin{picture}(1,1)(-1,-2)\circle*{2}\end{picture}\ }}}
\newcommand{\blank}{\underline{\hskip 10pt}}
\DeclareMathOperator{\trunc}{{trunc}}


%%*************Diagram of scheme notation
\DeclareMathOperator{\kSch}{{Sch}}
\DeclareMathOperator{\kCat}{{Cat}}
\DeclareMathOperator{\kHrc}{{Hrc}}
\newcommand{\uSch}{\underline{\kSch}}
\newcommand{\uCat}{\underline{\kCat}}
\newcommand{\uHrc}{\underline{\kHrc}}
\newcommand{\Ho}{\text{Ho}}

%%%*************Du Bois short hand******************
\newcommand{\DuBois}[1]{{\underline \Omega {}^0_{#1}}}
\newcommand{\FullDuBois}[1]{{\underline \Omega {}^{\mydot}_{#1}}}

\newcommand{\modGens}[2]{{{\bf \mu}_{#1}{\left(#2\right)}}}
\newcommand{\Sn}[1]{${\mathrm{S}_{#1}}$}
\newcommand{\Rn}[1]{${\mathrm{R}_{#1}}$}

%%%*************F-singularity short hand************
\newcommand{\tauCohomology}{T}
\newcommand{\FCohomology}{S}
\newcommand{\Ram}{\mathrm{Ram}}
\DeclareMathOperator{\Tr}{Tr}
\newcommand{\roundup}[1]{\lceil #1 \rceil}
\newcommand{\rounddown}[1]{\lfloor #1 \rfloor}
\newcommand{\HK}{\mathrm{HK}}

%%***************Latin short hand*******************
\newcommand{\cf}{{\itshape cf.} }
\newcommand{\loccit}{{\itshape loc. cit.} }
\newcommand{\ie}{{\itshape i.e.} }

%%************Script**********************
\newcommand{\sA}{\scr{A}}
\newcommand{\sB}{\scr{B}}
\newcommand{\sC}{\scr{C}}
\newcommand{\sD}{\scr{D}}
\newcommand{\sE}{\scr{E}}
\newcommand{\sF}{\scr{F}}
\newcommand{\sG}{\scr{G}}
\newcommand{\sH}{\scr{H}}
\newcommand{\sI}{\scr{I}}
\newcommand{\sJ}{\scr{J}}
\newcommand{\sK}{\scr{K}}
\newcommand{\sL}{\scr{L}}
\newcommand{\sM}{\scr{M}}
\newcommand{\sN}{\scr{N}}
\newcommand{\sO}{\scr{O}}
\newcommand{\sP}{\scr{P}}
\newcommand{\sQ}{\scr{Q}}
\newcommand{\sR}{\scr{R}}
\newcommand{\sS}{\scr{S}}
\newcommand{\sT}{\scr{T}}
\newcommand{\sU}{\scr{U}}
\newcommand{\sV}{\scr{V}}
\newcommand{\sW}{\scr{W}}
\newcommand{\sX}{\scr{X}}
\newcommand{\sY}{\scr{Y}}
\newcommand{\sZ}{\scr{Z}}

\newcommand{\enm}[1]{\ensuremath{#1}}
\newcommand{\mJ}{\mathcal{J}}
\newcommand{\uTwo}{\underline{2}}
\newcommand{\uOne}{\underline{1}}
\newcommand{\uZero}{\underline{0}}

\newcommand{\ba}{\mathfrak{a}}
\newcommand{\bb}{\mathfrak{b}}
\newcommand{\bc}{\mathfrak{c}}
\newcommand{\bff}{\mathfrak{f}}
\newcommand{\bm}{\mathfrak{m}}
\newcommand{\fram}{\mathfrak{m}}
\newcommand{\bn}{\mathfrak{n}}
\newcommand{\bp}{\mathfrak{p}}
\newcommand{\bq}{\mathfrak{q}}
\newcommand{\bt}{\mathfrak{t}}
\newcommand{\fra}{\mathfrak{a}}
\newcommand{\frb}{\mathfrak{b}}
\newcommand{\frc}{\mathfrak{c}}
\newcommand{\frf}{\mathfrak{f}}
\newcommand{\frm}{\mathfrak{m}}
%\renewcommand{\frm}{\mathfrak{m}}
\newcommand{\frn}{\mathfrak{n}}
\newcommand{\frp}{\mathfrak{p}}
\newcommand{\frq}{\mathfrak{q}}
\newcommand{\frt}{\mathfrak{t}}

\newcommand{\bA}{\mathbb{A}}
\newcommand{\bB}{\mathbb{B}}
\newcommand{\bC}{\mathbb{C}}
\newcommand{\bD}{\mathbb{D}}
\newcommand{\bE}{\mathbb{E}}
\newcommand{\bF}{\mathbb{F}}
\newcommand{\bG}{\mathbb{G}}
\newcommand{\bH}{\mathbb{H}}
\newcommand{\bI}{\mathbb{I}}
\newcommand{\bJ}{\mathbb{J}}
\newcommand{\bK}{\mathbb{K}}
\newcommand{\bL}{\mathbb{L}}
\newcommand{\bM}{\mathbb{M}}
\newcommand{\bN}{\mathbb{N}}
\newcommand{\bO}{\mathbb{O}}
\newcommand{\bP}{\mathbb{P}}
\newcommand{\bQ}{\mathbb{Q}}
\newcommand{\bR}{\mathbb{R}}
\newcommand{\bS}{\mathbb{S}}
\newcommand{\bT}{\mathbb{T}}
\newcommand{\bU}{\mathbb{U}}
\newcommand{\bV}{\mathbb{V}}
\newcommand{\bW}{\mathbb{W}}
\newcommand{\bX}{\mathbb{X}}
\newcommand{\bY}{\mathbb{Y}}
\newcommand{\bZ}{\mathbb{Z}}


\newcommand{\al}{\alpha}
\newcommand{\be}{\beta}
\newcommand{\ga}{\gamma}
\newcommand{\de}{\delta}
\newcommand{\pa}{\partial}   %pretend its Greek
\newcommand{\epz}{\varepsilon}
\newcommand{\ph}{\phi}
\newcommand{\phz}{\varphi}
\newcommand{\et}{\eta}
\newcommand{\io}{\iota}
\newcommand{\ka}{\kappa}
\newcommand{\la}{\lambda}
\newcommand{\tha}{\theta}
\newcommand{\thz}{\vartheta}
\newcommand{\rh}{\rho}
\newcommand{\si}{\sigma}
\newcommand{\ta}{\tau}
\newcommand{\ch}{\chi}
\newcommand{\ps}{\psi}
\newcommand{\ze}{\zeta}
\newcommand{\om}{\omega}
\newcommand{\GA}{\Gamma}
\newcommand{\LA}{\Lambda}
\newcommand{\DE}{\Delta}
\newcommand{\SI}{\Sigma}
\newcommand{\THA}{\Theta}
\newcommand{\OM}{\Omega}
\newcommand{\XI}{\Xi}
\newcommand{\UP}{\Upsilon}
\newcommand{\PI}{\Pi}
\newcommand{\PS}{\Psi}
\newcommand{\PH}{\Phi}

\newcommand{\com}{\circ}     % composition of functions
\newcommand{\iso}{\simeq}    % preferred isomorphism symbol
\newcommand{\ten}{\otimes}   % tensor product
\newcommand{\add}{\oplus}    % direct sum

\newcommand{\ul}{\underline}
\newcommand{\nsubset}{\not\subset}
\newcommand{\tld}{\widetilde }
\renewcommand{\:}{\colon}


\newcommand{\rtarr}{\longrightarrow}
\newcommand{\ltarr}{\longleftarrow}
\newcommand{\from}{\longleftarrow}
\newcommand{\monoto}{\lhook\joinrel\relbar\joinrel\rightarrow}
\newcommand{\epito}{\relbar\joinrel\twoheadrightarrow}

%%%%%%%%% math short hand
%%%% gothic
\newcommand{\Schs}{\mathfrak S\mathfrak c\mathfrak h_{S}}
\newcommand{\LocFrees}{\mathfrak L\mathfrak o\mathfrak c\mathfrak F\mathfrak
 r\mathfrak e\mathfrak e_{S}}
\newcommand{\A}{\mathfrak A}
\newcommand{\Ab}{\mathfrak A\mathfrak b}
\newcommand{\B}{\mathfrak B}
\newcommand{\M}{\mathfrak M\mathfrak o\mathfrak d}
\newcommand{\Mg}{\mathfrak M_g}
\newcommand{\Mgbar}{\overline{\mathfrak M}_g}
\newcommand{\Mh}{\mathfrak M_h}
\newcommand{\Mhbar}{\overline{\mathfrak M}_h}
\newcommand{\maxm}{\mathfrak m}

%%%% curly
\newcommand{\m}{\scr M}
\newcommand{\n}{\scr N}
\newcommand{\cO}{\mathcal O}
\renewcommand{\O}{\mathcal O}
\newcommand{\f}{\scr F}
\renewcommand{\O}{\scr O}
\newcommand{\I}{\scr I}
\newcommand{\J}{\scr{J}}

%%%% Blackboard bold
\newcommand{\C}{\mathbb {C}}
\newcommand{\N}{\mathbb {N}}
\newcommand{\R}{\mathbb {R}}
\newcommand{\PP}{\mathbb {P}}
\newcommand{\Z}{\mathbb {Z}}
\newcommand{\Q}{\mathbb {Q}}
\renewcommand{\r}{\mathbb R^{+}}
\newcommand{\NZ}{\mbox{$\mathbb{N}$}}
\renewcommand{\O}{\mbox{$\mathcal{O}$}}
\renewcommand{\P}{\mathbb{P}}
\newcommand{\ZZ}{\mbox{$\mathbb{Z}$}}
%%%%
\newcommand{\infinity}{\infty}
\newcommand{\ney}{\overline{NE}(Y)}
\newcommand{\nex}{\overline{NE}(X)}
\newcommand{\nes}{\overline{NE}(S)}
%%%%
\newcommand{\sub}{\subseteq}
\newcommand{\ratmap}{\dasharrow}
\newcommand{\eq}{\equiv}
\newcommand{\myquad}{\ }
%%%
%%%%%%% operators
\DeclareMathOperator{\Char}{{char}}
\DeclareMathOperator{\Cart}{{Cartier}}
\DeclareMathOperator{\fpt}{{fpt}}
\DeclareMathOperator{\divisor}{{div}}
\DeclareMathOperator{\Div}{{div}}
\DeclareMathOperator{\ord}{{ord}}
\DeclareMathOperator{\Frac}{{Frac}}
\DeclareMathOperator{\Ann}{{Ann}}
\DeclareMathOperator{\rd}{{rd}}
\DeclareMathOperator{\an}{{an}}
\DeclareMathOperator{\height}{{ht}}
\DeclareMathOperator{\exc}{{exc}}
\DeclareMathOperator{\coherent}{{coh}}
\DeclareMathOperator{\quasicoherent}{{qcoh}}
\DeclareMathOperator{\sn}{{sn}}
\DeclareMathOperator{\wn}{{wn}}
\DeclareMathOperator{\id}{{id}}
\DeclareMathOperator{\codim}{codim}
\DeclareMathOperator{\coker}{{coker}}
%%\DeclareMathOperator{\ker}{{ker}}
\DeclareMathOperator{\im}{{im}}
\DeclareMathOperator{\sgn}{{sgn}}
%%\DeclareMathOperator{\hom}{{Hom}}
\DeclareMathOperator{\opp}{{op}}
\DeclareMathOperator{\ext}{{Ext}}
\DeclareMathOperator{\Tor}{{Tor}}
\DeclareMathOperator{\pic}{{Pic}}
\DeclareMathOperator{\pico}{{Pic}^{\circ}}
\DeclareMathOperator{\aut}{{Aut}}
\DeclareMathOperator{\bir}{{Bir}}
\DeclareMathOperator{\lin}{{Lin}}
\DeclareMathOperator{\sym}{{Sym}}
\DeclareMathOperator{\rank}{{rank}}
\DeclareMathOperator{\rk}{{rk}}
\DeclareMathOperator{\pgl}{{PGL}}
\DeclareMathOperator{\gl}{{GL}}
\DeclareMathOperator{\Gr}{{Gr}}
\DeclareMathOperator{\ob}{{Ob}}
\DeclareMathOperator{\mor}{{Mor}}
\DeclareMathOperator{\supp}{{supp}}
\DeclareMathOperator{\Supp}{{Supp}}
\DeclareMathOperator{\Sing}{{Sing}}
\DeclareMathOperator{\var}{{Var}}
\DeclareMathOperator{\Spec}{{Spec}}
\DeclareMathOperator{\Proj}{{Proj}}
\DeclareMathOperator{\Tot}{{Tot}}
\DeclareMathOperator{\Aut}{Aut}
\DeclareMathOperator{\Lef}{Lef}
\DeclareMathOperator{\wt}{wt}
\DeclareMathOperator{\twoRC}{{RC_2^n}}
\DeclareMathOperator{\ptRC}{{RC_{\bullet}}}
\DeclareMathOperator{\twoptRC}{{RC^2_{\bullet}}}
\DeclareMathOperator{\Univ}{Univ}
\DeclareMathOperator{\Univrc}{{Univ^{rc}}}
\DeclareMathOperator{\twoUnivrc}{{Univ^{rc, 2}}}
\DeclareMathOperator{\ptUnivrc}{{Univ^{rc}_{\bullet}}}
\DeclareMathOperator{\twoptUnivrc}{{Univ_{\bullet}^{rc, 2}}}
\DeclareMathOperator{\charact}{char}
\DeclareMathOperator{\Chow}{Chow}
\DeclareMathOperator{\Dubbies}{Dubbies^n}
\DeclareMathOperator{\Ext}{Ext}
\DeclareMathOperator{\Hilb}{Hilb}
\DeclareMathOperator{\Hom}{Hom}
\DeclareMathOperator{\sHom}{{\sH}om}
\DeclareMathOperator{\Hombir}{Hom_{bir}^n}
\DeclareMathOperator{\Image}{Image}
\DeclareMathOperator{\genus}{genus}
\DeclareMathOperator{\Imaginary}{Im}
\DeclareMathOperator{\Img}{Im}
\DeclareMathOperator{\Ker}{Ker}
\DeclareMathOperator{\locus}{locus}
\DeclareMathOperator{\Num}{Num}
\DeclareMathOperator{\Pic}{Pic}
\DeclareMathOperator{\RatCurves}{RatCurves^n}
\DeclareMathOperator{\RC}{RatCurves^n}
\DeclareMathOperator{\red}{red}
\DeclareMathOperator{\reduced}{red}
\DeclareMathOperator{\Reg}{Reg}
\DeclareMathOperator{\psl}{PGL}
\DeclareMathOperator{\Sym}{Sym}
\DeclareMathOperator{\mult}{mult}
\DeclareMathOperator{\mld}{mld}
\renewcommand{\mod}[1]{\,(\textnormal{mod}\,#1)}
%%%%%%%%%%%%%%%%%%%%%%%%%%%%%%%%%%%%%
\def\spec#1.#2.{{\bold S\bold p\bold e\bold c}_{#1}#2}
\def\proj#1.#2.{{\bold P\bold r\bold o\bold j}_{#1}\sum #2}
\def\ring#1.{\scr O_{#1}}
\def\map#1.#2.{#1 \to #2}
\def\longmap#1.#2.{#1 \longrightarrow #2}
\def\factor#1.#2.{\left. \raise 2pt\hbox{$#1$} \right/
\hskip -2pt\raise -2pt\hbox{$#2$}}
\def\pe#1.{\mathbb P(#1)}
\def\pr#1.{\mathbb P^{#1}}
\newcommand{\sheafspec}{\mbox{\bf Spec}}
\newcommand{\len}[2]{{{\bf \ell}_{#1}{\left(#2\right)}}}

%%%%%%%%%%%%%%%%%%%%%%%%%%%%%%%%%%%%%%%%%%%%%%%%%
%%%%%% cohomology and short exact sequences %%%%%
%%%%%%%%%%%%%%%%%%%%%%%%%%%%%%%%%%%%%%%%%%%%%%%%%
\def\coh#1.#2.#3.{H^{#1}(#2,#3)}
\def\dimcoh#1.#2.#3.{h^{#1}(#2,#3)}
\def\hypcoh#1.#2.#3.{\mathbb H_{\vphantom{l}}^{#1}(#2,#3)}
\def\loccoh#1.#2.#3.#4.{H^{#1}_{#2}(#3,#4)}
\def\dimloccoh#1.#2.#3.#4.{h^{#1}_{#2}(#3,#4)}
\def\lochypcoh#1.#2.#3.#4.{\mathbb H^{#1}_{#2}(#3,#4)}
%%%%%%%%%%
\def\ses#1.#2.#3.{0  \longrightarrow  #1   \longrightarrow
 #2 \longrightarrow #3 \longrightarrow 0}
\def\sesshort#1.#2.#3.{0
 \rightarrow #1 \rightarrow #2 \rightarrow #3 \rightarrow 0}
 \def\sesa#1{0
 \rightarrow #1 \rightarrow #1 \rightarrow #1 \rightarrow 0}

\renewcommand{\to}[1][]{\xrightarrow{\ #1\ }}
\newcommand{\onto}[1][]{\protect{\xrightarrow{\ #1\ }\hspace{-0.8em}\rightarrow}}
\newcommand{\into}[1][]{\lhook \joinrel \xrightarrow{\ #1\ }}
%%%%%%%%%%
%%%%%%%%%% iff
\def\iff#1#2#3{
    \hfil\hbox{\hsize =#1 \vtop{\noin #2} \hskip.5cm
    \lower.5\baselineskip\hbox{$\Leftrightarrow$}\hskip.5cm
    \vtop{\noin #3}}\hfil\medskip}
%%%%%%%%%%%%%%%%%%%%%%%%%%%%%%
\def\myoplus#1.#2.{\underset #1 \to {\overset #2 \to \oplus}}
\def\assign{\longmapsto}
%%%%%%%%%%%%%%%%%%%%%%%%%%%%%%
%%%%%%%%%%%%%%%%%%%%%%%%%%%%%%%%%%
%%% Arrows %%%%%%%%%%%%%%
%%%%%%%%%%%%%%%%%%%%%%%%%%%%%%%%%%%


%\usepackage{pdfsync}
%
%

\newtheorem{theorem}{Theorem}[section]
\newtheorem{lemma}[theorem]{Lemma}
\newtheorem{proposition}[theorem]{Proposition}
\newtheorem{corollary}[theorem]{Corollary}
\newtheorem{conjecture}[theorem]{Conjecture}
\newtheorem{claim}[equation]{Claim}
\newtheorem*{mainthm}{Main Theorem}

\theoremstyle{definition}
\newtheorem{definition}[theorem]{Definition}
\newtheorem{example}[theorem]{Example}
\newtheorem{setting}[theorem]{Setting}
\newtheorem{xca}[theorem]{Exercise}
\newtheorem{exercise}[theorem]{Exercise}

\theoremstyle{remark}
\newtheorem{remark}[theorem]{Remark}
\newtheorem{hint}[theorem]{Hint}
\newtheorem{question}[theorem]{Question}
\newtheorem*{strategy*}{Strategy}


\DeclareMathOperator{\Tr}{Tr}
\DeclareMathOperator{\Image}{Image}
\DeclareMathOperator{\Div}{div}
\renewcommand{\O}{\mathcal O}
\newcommand{\bQ}{\mathbb{Q}}
\newcommand{\myR}{\mathbf{R}}
\newcommand{\bm}{\mathfrak{m}}
\DeclareMathOperator{\Hom}{Hom}
\DeclareMathOperator{\sHom}{\mathcal{H}om}
\DeclareMathOperator{\Spec}{Spec}
\newcommand{\bF}{\mathbf{F}}

\begin{document}

\title{A brief tutorial on the F-singularities package, version 0.1}
\author{Package by:  Mordechai Katzman, Sara Malec, Karl Schwede, Emily Witt}

\maketitle

This is a package for use in Macaulay2 for doing computations of $F$-pure thresholds, test ideals, non-sharply $F$-pure ideals, and related notions.  This document gives a brief tutorial on how to use the package.

\section{Loading the package}

This package has been developed using Macaulay2 version 1.6.  Use other versions at your own risk.

Place the package {\tt PosChar.m2} in a folder that Macaulay2 can see (for example, the folder from which you are starting Macaulay2 or emacs), and run the command \begin{center}{\tt loadPackage "PosChar"}  \end{center}
If there are no errors, then it worked.

If you would like to automatically load the package on startup, please edit your {\tt .Macaulay2/init.m2} and add a line such as
\begin{quotation}
{\tt if fileExists "/home/myUserName/F-sing/PosChar.m2" then path=append(path, "/home/myUserName/F-sing/")}\\
{\tt  if fileExists "/home/myUserName/F-sing/PosChar.m2" then loadPackage("PosChar")}
\end{quotation}
You should edit your path appropriately.


\section{$F$-pure thresholds}

The default function here is {\tt estFPT}.  To use it, create a polynomial ring over a finite field, say {\tt R = ZZ/5[x,y]} and then create an element, say ${\tt f  = x^2 - y^3 }$.  Then running {\tt estFPT(f, 1)} will try to compute the $F$-pure threshold (if it fails, it should give a range in which the FPT can be found, or it crash due to lack of RAM).  You can replace $1$ by any integer {\tt e}, larger integers {\tt e} may result in longer running time, but perhaps also more success (additional details as to where {\tt e} is used can be found below).

It does it in the following way.  (To watch the progress, run {\tt estFPT} with the {\tt Verbose=>true} switch).
\begin{enumerate}
\item It first checks whether or not the polynomial {\tt f} is diagonal (using the function {\tt isDiagonal}).  If so, it computes the FPT using the methods from D.~Hern\'andez's thesis, by calling the function {\tt diagonalFPT}.  You can turn this check off by including the parameter inside the function call, {\tt DiagonalCheck=>false}.  {\tt diagonalFPT} is maintained by Emily Witt.
\item Next it checks whether or not the polynomial {\tt f} is binomial (using the function {\tt isBinomial}).  If so, it computes the FPT using the methods from D.~Hern\'andez's thesis, by calling the function {\tt binomialFPT}.  You can turn this check off by including the parameter inside the function call, {\tt BinomialCheck=>false}.  {\tt binomialFPT} is maintained by Emily Witt.
\item If these operations fail, using the {\tt e} you provided, it will compute the largest $\nu_e$ such that ${\tt f^{\nu_e} \notin \langle x_1, \ldots, x_n \rangle}$.  To compute this ${\tt \nu_e}$ yourself, use the functions {\tt nu} and {\tt nuList} (maintained by Emily Witt).   Regardless, we now know that ${ \tt \nu_e \over p^e - 1}\leq \mathrm{FPT}({\tt f}) \leq { \tt \nu_e + 1 \over p^e}$.
\item We check whether ${\tt (R, f^{\nu_e \over p^e - 1})}$ is strongly $F$-regular by calling the function {\tt isFRegularPoly}, if it is not, then ${ \tt \nu_e \over p^e - 1} = \mathrm{FPT}({\tt f})$.  Indeed, if the denominator of the FPT is not divisible by $p > 0$, then this method will always return the FPT as long as you provide a sufficiently divisible {\tt e}.  You can turn this check off by including the parameter inside the function call, {\tt NuPEMinus1Check=>false}.  This part of the function and {\tt isFRegularPoly} is maintained by Karl Schwede.
\item If the previous check failed, then the $F$-signature of ${\tt (R, f^{\nu_e \over p^e})}$ and ${\tt (R, f^{\nu_e - 1 \over p^e})}$ are computed using the {\tt fSig} function.  This can be very slow, but currently it cannot be turned off.  You can try to run this in a multithreaded way with {\tt MultiThread=>true}, but it provides no performance gains at the moment.  Substantial performance gains can be found by using a different monomial order, or making sure your polynomial is quasi-homogeneous.  Regardless, the secant line between
    \begin{center}
${\tt \Big( {\nu_e \over p^e}, (R, f^{\nu_e \over p^e}) \Big) } \text{ and } {\tt \Big( {\nu_e - 1 \over p^e}, (R, f^{\nu_e - 1 \over p^e}) \Big)}$
    \end{center}
    intersects the $x$-axis at a point ${\tt a} \leq  \mathrm{FPT}({\tt f})$.  This part is maintained by Karl Schwede.
\item Finally, the program checks whether ${\tt (R, f^{a})}$ is strongly $F$-regular again using the function {\tt isFRegularPoly}.    If not, then $a = \mathrm{FPT}({\tt f})$.  Otherwise, the range ${\tt [a, {\nu_e + 1 \over p^e}]}$ is returned.  This method may never find the FPT, even for large and divisible $e$.
\end{enumerate}

In the future, we hope to be able to compute FPTs for wider classes of quasi-homogeneous polynomials, for non-principal ideals, and for more general ambient rings.  If you need some of this functionality \emph{now}, please contact us and we may be able to help.

\section{Test ideals and non-sharply-$F$-pure ideals}

There are several functions for computing test ideals.  We list them below.  These are maintained by Karl Schwede, but they rely heavily on the {\tt ethRoot} function (which computes $\bullet^{[1/p^e]}$ also denoted by $I_e(\bullet)$) written and maintained by Mordechai Katzman.

\begin{enumerate}
\item {\tt tauPoly(f,t)}  Suppose ${\tt R}$ is a polynomial ring containing an element {\tt f} and ${\tt t \geq 0}$ is a rational number.  This computes the test ideal ${\tt \tau(R, f^t)}$.  This is done by writing ${\tt t = {a \over (p^b - 1)p^c}}$, first computing ${\tt \tau(R, f^{a\over p^b-1})}$ via ${\tt tauAOverPEMinus1Poly(f,a,b)}$.  And then using the formula
    \begin{center}
    ${\tt \tau(R, f^{a\over p^b-1})^{[1/p^c]} = \tau(R, f^{a\over (p^b-1)p^c})}$.
    \end{center}
    The ${\tt [1/p^c]}$ implementation was originally written by Katzman.
\item {\tt tauAOverPEMinus1Poly(f,a,e)}  Suppose ${\tt R}$ is a polynomial ring containing an element {\tt f} and ${\tt a,e \geq 1}$ are integers.  This computes the test ideal ${\tt \tau(R, f^{a \over p^e -1})}$.  This uses the same strategy as the work of Katzman for computing parameter test ideals.
\item {\tt tauQGor(R, e, f, t)}  Suppose that ${\tt R}$ is a $\bQ$-Gorenstein normal ring containing an element ${\tt f}$ and that the index of $K_{\tt R}$ divides ${\tt p^e - 1}$.  Further $t \geq 0$ is a rational number.  Then this computes the test ideal ${\tt \tau(R, f^t)}$.  The strategy is similar to {\tt tauPoly(f,t)} above.
\item {\tt tauQGorAmb(R, e)}  Suppose that ${\tt R}$ is a $\bQ$-Gorenstein normal ring and that the index of $K_{\tt R}$ divides ${\tt p^e - 1}$.  Then this computes the test ideal ${\tt \tau(R)}$.
\end{enumerate}

Using these functions, we also have implemented $F$-regularity checks:  for polynomial rings {\tt isFRegularPoly} and for $\bQ$-Gorenstein rings {\tt isFRegularQGor}.

There is also support for non-sharply-$F$-pure ideals.
\begin{enumerate}
\item {\tt sigmaAOverPEMinus1Poly(f, a, e)} Suppose ${\tt R}$ is a polynomial ring containing an element {\tt f} and ${\tt a,e \geq 1}$ are integers.  This computes the non-sharply-$F$-pure ideal ${\tt \sigma(R, f^{a \over p^e - 1})}$.
\item {\tt sigmaQGorAmb(R, g)}  Suppose that ${\tt R}$ is a $\bQ$-Gorenstein normal ring and that the index of $K_{\tt R}$ divides ${\tt p^g - 1}$.  Then this computes the non-sharply-$F$-pure ideal ${\tt \sigma(R)}$.
\item {\tt sigmaQGorAOverPEMinus1(f, a, e, g)}  Suppose that ${\tt R}$ is a $\bQ$-Gorenstein normal ring containing an element ${\tt f}$ and that the index of $K_{\tt R}$ divides ${\tt p^g - 1}$.  Further ${\tt a,e \geq 1}$ are integers.  Then this computes the non-sharply-$F$-pure ideal ${\tt \sigma(R, f^{a \over p^e - 1})}$.
\end{enumerate}
In each of these cases, $\sigma$ is the stable image of certain $p^{-e}$-linear maps, which can be reduced to twists of {\tt ethRoot}.  Using these functions, we also have implemented a sharp $F$-purity check for polynomial rings {\tt isSharplyFPurePoly}.

In the future we also hope to include parameter test module and ideal computations and $F$-rationality / $F$-injectivity checks.  These have previously been implemented by Katzman.  We also hope to include within this package tools for computing compatibly split ideals, and more generally $\phi$-fixed ideals.  See the recent work of Boix-Katzman (and also Katzman-Zhang).

\end{document}

